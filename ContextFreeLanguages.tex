%&cf
% %PRECOMPILE COMMAND pdftex -ini -jobname="plurals_event_chap_34" "&pdflatex" mylatexformat.ltx plurals_event_chap_34.tex
% \documentclass[english]{article}
% \usepackage[T1]{fontenc}
% \usepackage[latin1]{inputenc}
% \usepackage[sort]{natbib}
% \usepackage{stdprmbl}

% \usepackage{parskip}

% \lingset{interpartskip = -5pt}
% \lingset{aboveexskip = 3pt,belowexskip = 3pt}


% \usepackage{listings}

% \definecolor{gray_ulisses}{gray}{0.55}
% \definecolor{castanho_ulisses}{rgb}{0.71,0.33,0.14}
% \definecolor{preto_ulisses}{rgb}{0.41,0.20,0.04}
% \definecolor{green_ulises}{rgb}{0.2,0.75,0}

% \lstdefinelanguage{HaskellUlisses} {
% 	basicstyle=\footnotesize,
% 	sensitive=true,
% 	morecomment=[l][\color{gray_ulisses}\ttfamily\tiny]{--},
% 	morecomment=[s][\color{gray_ulisses}\ttfamily\tiny]{\{-}{-\}},
% 	morestring=[b]",
% 	stringstyle=\color{red},
% 	showstringspaces=false,
% 	numberstyle=\tiny,
% 	numberblanklines=true,
% 	showspaces=false,
% 	breaklines=true,
% 	showtabs=false,
% 	emph=
% 	{[1]
% 		FilePath,IOError,abs,acos,acosh,all,and,any,appendFile,approxRational,asTypeOf,asin,
% 		asinh,atan,atan2,atanh,basicIORun,break,catch,ceiling,chr,compare,concat,concatMap,
% 		const,cos,cosh,curry,cycle,decodeFloat,denominator,digitToInt,div,divMod,drop,
% 		dropWhile,either,elem,encodeFloat,enumFrom,enumFromThen,enumFromThenTo,enumFromTo,
% 		error,even,exp,exponent,fail,filter,flip,floatDigits,floatRadix,floatRange,floor,
% 		fmap,foldl,foldl1,foldr,foldr1,fromDouble,fromEnum,fromInt,fromInteger,fromIntegral,
% 		fromRational,fst,gcd,getChar,getContents,getLine,head,id,inRange,index,init,intToDigit,
% 		interact,ioError,isAlpha,isAlphaNum,isAscii,isControl,isDenormalized,isDigit,isHexDigit,
% 		isIEEE,isInfinite,isLower,isNaN,isNegativeZero,isOctDigit,isPrint,isSpace,isUpper,iterate,
% 		last,lcm,length,lex,lexDigits,lexLitChar,lines,log,logBase,lookup,map,mapM,mapM_,max,
% 		maxBound,maximum,maybe,min,minBound,minimum,mod,negate,not,notElem,null,numerator,odd,
% 		or,ord,otherwise,pi,pred,primExitWith,print,product,properFraction,putChar,putStr,putStrLn,quot,
% 		quotRem,range,rangeSize,read,readDec,readFile,readFloat,readHex,readIO,readInt,readList,readLitChar,
% 		readLn,readOct,readParen,readSigned,reads,readsPrec,realToFrac,recip,rem,repeat,replicate,return,
% 		reverse,round,scaleFloat,scanl,scanl1,scanr,scanr1,seq,sequence,sequence_,show,showChar,showInt,
% 		showList,showLitChar,showParen,showSigned,showString,shows,showsPrec,significand,signum,sin,
% 		sinh,snd,span,splitAt,sqrt,subtract,succ,sum,tail,take,takeWhile,tan,tanh,threadToIOResult,toEnum,
% 		toInt,toInteger,toLower,toRational,toUpper,truncate,uncurry,undefined,unlines,until,unwords,unzip,
% 		unzip3,userError,words,writeFile,zip,zip3,zipWith,zipWith3,listArray,doParse
% 	},
% 	emphstyle={[1]\color{blue}},
% 	emph=
% 	{[2]
% 		Bool,Char,Double,Either,Float,IO,Integer,Int,Maybe,Ordering,Rational,Ratio,ReadS,ShowS,String,
% 		Word8,InPacket, ContextFree, ContextFreeRelation
% 	},
% 	emphstyle={[2]\color{green!40!black}},
% 	emph=
% 	{[3]
% 		case,class,data,deriving,do,else,if,import,in,infixl,infixr,instance,let,
% 		module,of,primitive,then,type,where
% 	},
% 	emphstyle={[3]\color{blue}\textbf},
% 	emph=
% 	{[4]
% 		quot,rem,div,mod,elem,notElem,seq, pure, mappend, mempty
% 	},
% 	emphstyle={[4]\color{purple!40!black}\textbf},
% 	emph=
% 	{[5]
% 		EQ,False,GT,Just,LT,Left,Nothing,Right,True,Show,Eq,Ord,Num, Monoid, Collection, Applicative
% 	},
% 	emphstyle={[5]\color{green!40!black}\textbf}
% }

% \lstset{xleftmargin=.50in, language=HaskellUlisses}

% \usepackage{amsthm}
% \newtheorem{prop}{Proposition}
% \newtheorem{definition}{Definition}

% \NewEnviron{nothing}{}
\endofdump

\title{Context-free languages: generalizing \& characterizing}
\author{Keny Chatain}

\begin{document}
\maketitle

In this note, I propose to characterize a constructive characterization of the class of context-free languages. The goal is to have a characterization similar to the characteriation of the class of rational languages, as the class of smallest languages closed under some natural operations.

The formal details are interspersed with a Haskell implementation.

\section{The word setting}

\subsection{Introduction}

Here, we assume an alphabet $\Sigma$ (\textsf{Char} in the Haskell implementation) and the set of strings on that alphabet, $\Sigma^*$ (\textsf{String}).

A language is a collection of strings. The use of the word \quo{collection}, instead of \quo{set}, is meant to reflect the fact that not all sets operations will be available to us. A collection, only makes available some operations like union and singleton set and certain form of set comprehension (the latter two are properties of \textsf{Applicative}): 

\lstinputlisting[firstline = 6, lastline = 14]{ContextFree.hs}

Using an applicative allows us to \quo{lift} any operation and any element defined on words into the realm of languages. To an element, corresponds the singleton language of that element. To word concatenation, corresponds language concatenation:

\lstinputlisting[firstline = 16, lastline = 17]{ContextFree.hs}

So languages (i.e. collection of words) have two types of operations defined on them: those that can be imported from the underlying structure of words (concatenate, elements) and those that follow from the structure of a collection (e.g. union). With this, one may define some finite languages:

\lstinputlisting[firstline = 22, lastline = 28]{ContextFree.hs}

In fact, taking advantage of recursivity/lazy evaluation, we can also define infinite languages as well:

\lstinputlisting[firstline = 30, lastline = 32]{ContextFree.hs}

More generally, Kleene star can be defined through a form of recursivity as well:

\lstinputlisting[firstline = 34, lastline = 36]{ContextFree.hs}

So the class of languages we can define is also closed under union, concatenation and Kleene star and contains all finite sets. A minima then, we can define any regular language:

\lstinputlisting[firstline = 38, lastline = 40]{ContextFree.hs}

But recursivity allows us to define more languages than just that. For instance, the famed non-regular $\left\lbrace a^nb^n\ \middle|\ n\geq 0\right\rbrace$:

\lstinputlisting[firstline = 42, lastline = 44]{ContextFree.hs}

In fact, we can \emph{build} a context-free grammar, by making use of crossed recursivity:

\lstinputlisting[firstline = 46, lastline = 55]{ContextFree.hs}

So all context-free languages can be defined in Haskell. Can we do even more than that? Interestingly no. The class of languages definable in Haskell, using only the operations made available by collections, words and recursivity, is the class of context-free languages. In some sense, context-free languages is the smallest class of languages closed under union and concatenation and recursive functions made from them.

In the next section, I set out to prove that result in its formal details.

\subsection{Formalization}

Before we do so, the notion of recursivity needs to be formalized. Informally, recursivity is a way of generating fixed points.

\lstinputlisting[firstline = 57, lastline = 65]{ContextFree.hs}

Of course, in Haskell, these is no guarantee of convergence, hence no guarantee that there is a fixed-point to any function. However, in the limited realm of functions that we consider, such guarantees are possible:

\begin{prop}
If $f$ is an \emph{increasing} function from languages to languages (i.e. $L\subset L' \Rightarrow f(L)\subset f(L')$), then 
$$\mathbf{FixPt}(f) = \bigcap \left\lbrace L\ \middle|\ f(L)\subset L\right\rbrace$$
is a fixed point. It is the smallest fixed point in fact (for subsethood).
\end{prop}
%
Coincidentally, all the functions we can define with union and concatenation are increasing. Had we been working with sets, instead of collections, and allowed such operations as complementation, we could construct non-monotonic function and the fixed-point guarantee would have vanished. A further thing to note is that in some high-order sense, $\mathbf{FixPt}$ is increasing:

\begin{prop}
If $f$ and $f'$ are two increasing functions, and $f'$ dominates $f$ (i.e. $f(L)\subset f'(L)$ for all $L$), then 
$$\mathbf{FixPt}(f) \subset \mathbf{FixPt}(f')$$
\end{prop}
%
So all the operations at our disposal - union, concatenation- will only ever generate increasing functions. Any function formed from \textbf{FixPt} will also be increasing and \textbf{FixedPt} can apply to it. So it makes sense to talk about the smallest set containing finite languages closed under these 3 operations: union, concatenation, and fixed point. The next section makes that precise.

\subsection{Context-free expressions}

We are aiming for a characterization of context-free languages in terms of some primitive elements and operations, the same way rational languages are defined as the smallest class of languages containing singletons and closed under rational operations (union, concatenation, Kleene star). The difference is that here, our primitive operations include \textbf{FixPt}, which operates on functions. Therefore, our expressions will need to model both languages and functions on languages at the same time.

\begin{definition}
A context-free expression is:
\begin{itemize}
\item a word (e.g. $a$, $abba$) (type \textsf{ContextFree})
\item a variable over languages (type \textsf{ContextFree})
\item the union of two context-free expressions of type \textsf{ContextFree},  denoted $L + L'$
\item the concatenation of two context-free expressions of type \textsf{ContextFree}, denoted $LL'$
\item the least fixed-point of a context-free function (type \textsf{ContextFree}$\rightarrow$\textsf{ContextFree}), denoted $\mathbf{Fix}(L)$
\item if $E$ is a context-free expression of type $a$, $\lambda X. E$ is an expression of type $\textsf{ContextFree}\rightarrow a$
\item if $E$ is a context-free expression of type $a\rightarrow b$ and $E'$ an expression of type $a$, $E(E')$ is an expression of type $b$
\end{itemize}
\end{definition}

We can provide a straightforward semantics for these expressions\footnote{Because of variables, I have to define rational expressions syntactically and then provide a semantics for these expressions. To avoid this extra step, we could use variable-free representations using combinators from e.g. combinatory logic. However, what we gain in conceptual transparency, we lose in notational transparency: the simplest function require a lot of combinators to write.}:

\begin{definition}
The object denoted by a context-free expression S
\begin{itemize}
\item $\dbbm{w}^{g} = \left\lbrace w\right\rbrace$
\item $\dbbm{X}^{g} = g(X)$
\item $\dbbm{L + L'}^{g} = \dbbm{L}^{g} \cup\dbbm{L'}^{g} $
\item $\dbbm{LL'}^{g} = \dbbm{L}^{g} \dbbm{L'}^{g} $
\item $\dbbm{\lambda X.\ E}^{g} = L\mapsto \dbb{E}^{g[X\leftarrow L]} $
\item $\dbbm{E(E')}^{g} = \dbbm{E}^{g}\left( \dbbm{E'}^{g}\right) $
\item $\dbbm{\mathbf{Fix(E)}}^{g} = \mathbf{FixPt}(\dbbm{\mathbf{E}}^{g}) $
\end{itemize}
\end{definition}
%
Because of the type system and the facts about increasingness discussed in the last subsection, all of these operations are going to be well-defined.

\subsection{Equivalence to context-free languages}

The context-free languages are those languages denoted by a context-free expression. To prove this, two steps are needed. We need to be able to convert from a context-free grammar to context-free expression and vice-versa.

\paragraph*{Obtaining a context-free grammar from a context-free expression.} To each expression $E$, we associate a set of context-free rewrite rules \textbf{Rules(E)} and a distinguished start symbol \textbf{Start(E)} as below. I assume that to all language variables correspond a non-terminal and $S_i$ will stand for any nonterminal that hasn't be used so far.

\begin{itemize}
\item 
$\mathbf{Rules}[w] = [S\rightarrow w]$\\
$\mathbf{Start}[w] = S$
\item 
$\mathbf{Rules}[X] = []$\\
$\mathbf{Start}[X] = X$
\item 
$\mathbf{Rules}[EE'] = [S \rightarrow \mathbf{Start}(E)\mathbf{Start}(E')]$\\
$\mathbf{Start}[EE'] = S$
\item 
$\mathbf{Rules}[E + E'] = [S \rightarrow \mathbf{Start}(E), S\rightarrow\mathbf{Start}(E')]$\\
$\mathbf{Start}[E + E'] = S$
\item 
$\mathbf{Rules}[\lambda X. E] = \mathbf{Rules}(E)$\\
$\mathbf{Start}[\lambda X. E] = \mathbf{Start}(E)$
\item 
$\mathbf{Rules}[(\lambda X. E)E'] = \mathbf{Rules}(E) +\hspace{-3pt}+\ \mathbf{Rules}(E') +\hspace{-3pt}+\ [X\rightarrow \textbf{Start}(E')]$\\
$\mathbf{Start}[(\lambda X. E)E'] = \mathbf{Start}(E)$
\item 
$\mathbf{Rules}[\mathbf{Fix}(\lambda X. E)] = \mathbf{Rules}(E) +\hspace{-3pt}+\ [X\rightarrow \textbf{Start}(E)]$\\
$\mathbf{Start}[\mathbf{Fix}(\lambda X. E)] = X$
\end{itemize}

\paragraph*{Obtaining a context-free expression from a context-free grammar.} The following is extremely reminiscent of the method used to resolve system of language equations for regular expressions (cf proof of Kleene's theorem). The first step is to rewrite a context free grammar as a system of equations over languages. I will show how this is done on a simple example ; hopefully, this example will make a more general statement superfluous. To each terminal $X$ corresponds the unknown $L_X$

\begin{figure}
\begin{tabularx}{\textwidth}{X  X  X}
Original grammar & Factorizing left-hand sides & Convert to a system of equation\\
\midrule
$T \rightarrow aVb$\newline
$V \rightarrow \epsilon$ \newline
$V \rightarrow T c V$
&
$T \rightarrow aVb$\newline
$V \rightarrow \epsilon + T c V$
&
$L_T = aL_Vb$\newline
$L_V = \epsilon + L_T c L_V$
\end{tabularx}
\end{figure}
%
There is of course a relation between the solution of this system of equations and the context-free language itself:

\begin{prop}\label{prop:sys_eqn}
The system of equations obtained from a context-free grammar has a smallest solution (for inclusion). This smallest solution is the tuple $(L_{X_1} = L(X_1),\ldots, L_{X_n}= L(X_n))$ where $L(X_i)$ is the language generated by non-terminal $X_i$ 
\end{prop}
%
So in our example:, we can in particular write

$$
\begin{array}{lcl}
L(T) &=& aL(V)b\\
L(V) &=& \epsilon + L(T) c L(V)
\end{array}
$$

Each of these equation can be seen as stating that a certain $L(X)$ is the fixed point of a function\footnote{Note that I am temporarily suspending the distinction between context-free expressions and the language they denote. This distinction is a technical hassle.}:

$$L(V) = \left( 	\lambda X.\ \epsilon + L(T)cX\right)\left( L(V)\right)$$

Two things: first, each $L(X)$ has to be the smallest such fixed point. Otherwise, we could form a non-bigger solution by replacing $L(X)$ with a non-bigger fixed point in the solution tuple. Second, the function is increasing that each $L(X)$ is the fixed point of is increasing. In other words:

$$L(V) = \mathbf{Fix}\left( \lambda X.\ \epsilon + L(T)cX\right)$$

These remarks allow us to reduce any of the $L(X)$ into a context-free expression. The idea is the same as with any system of equations: express one unknown in terms of the other, replace the unknown with its expression in terms of the other in all the other equations, repeat until all unknowns are eliminated. So in our particular case, we can replace $L(V)$ with its expression above in the first equation of the system

$$L(T) = a\mathbf{Fix}\left( \lambda X.\ \epsilon + L(T)cX\right)b$$

This can be taken to mean that $L(T)$ is the fixed point of the function:

$$L(T) = \left(\lambda Y.\ a\mathbf{Fix}\left( \lambda X.\ \epsilon + YcX\right)b\right)\left(L(T)\right)$$

With a similar reasoning as above, we can conclude that:

$$L(T) = \mathbf{Fix}\left(\lambda Y.\ a\mathbf{Fix}\left( \lambda X.\ \epsilon + YcX\right)b\right)$$

If $T$ is the start symbol of the grammar, then we just found that the context-free language can be expressed by a context-free expression. This concludes the proof that every context-free language is expressible in terms of these expressions


\subsection{Missing pieces}

There is one aspect of the Haskellization which is regrettable. In Haskell, there is a way to write context-sensitive languages if we allow ourselves to lift more complex string-based operations than the primitive operation of word, concatenation. For instance, this implementation of $\left\lbrace a^nba^nba^n\ \middle|\ n\geq 0\right\rbrace$:

\lstinputlisting[firstline = 68, lastline = 71]{ContextFree.hs}

Added to that, it can be argued that the Haskell fragment is more expressive than context-free expressions. For instance, Haskell can express functions of type $(\textsf{ContextFree}\rightarrow \textsf{ContextFree})\rightarrow  \textsf{ContextFree}\rightarrow \textsf{ContextFree}$, while context-free expressions do not. If such functions are available to Haskell, can Haskell express more languages than context-free expressions can? 

Formally, this corresponds to asking whether we would obtain non-context-free languages if we allowed higher-order fixed-points, i.e. fixed points of $(\textsf{ContextFree}\rightarrow \textsf{ContextFree})\rightarrow  \textsf{ContextFree}\rightarrow \textsf{ContextFree}$. We would need a theorem giving us the existence of a privileged fixed point for such expressions. There are such guarantees

Contrary to my own expectations, it turns out that context-sensitive languages can be expressed using higher-order fixed-points:

\lstinputlisting[firstline = 73, lastline = 79]{ContextFree.hs}



\begin{nothing}
\subsection{Generalization}
If $(E, \prec)$ is an ordered set where every subset has an infimum and a supremum, then there is an ordering on $E^F$ with the same properties.

\begin{proof}
The order is domination. $f'$ dominates $f$ iff for every $x$, $f(x)\prec f'(x)$. We write $f\prec_D f'$. Then the supremum of a set of function is obtained as the pointwise supremum

$\sup \mathcal F = \lambda x.\ \sup\left\lbrace f(x)\ \middle|\ f\in \mathcal{F}\right\rbrace$

This is indeed a supremum. If $g$ dominates all functions in $\mathcal F$, then

$$\forall x,\forall f\in\mathcal{F},\ f(x)\prec g(x)$$
From which it follows that:

$$\forall x,\ \sup \mathcal F(x) = \sup\left\lbrace f(x)\ \middle|\ f\in \mathcal{F}\right\rbrace\prec g(x)$$

So $\sup \mathcal F$ is an upper bound. Now, let $g$ be any function that does not dominate $\sup \mathcal{F}$. That is to say:

$$\exists x,\ \sup \mathcal{F}(x)\not\prec g(x) $$

In other words:

$$\exists x,\ \sup\left\lbrace f(x)\ \middle|\ f\in \mathcal{F}\right\rbrace\not\prec g(x) $$

For this this to be so, there has to be some value $f(x)$ that $g(x)$ is not greater than, by definition of the supremum:

$$\exists x,\exists f,\ f(x)\not\prec g(x) $$

So $g$ does not dominate all functions in $\mathcal F$; it is not an upper bound of $\mathcal F$

$$\exists f,\ f\not\prec_D g $$

By contraposition, this means that every upper bound of $\mathcal F$ dominate $\sup \mathcal F$, making it the least upper bound, the infimum.
A similar proof is available for the infimum
\end{proof}

If $(E, \prec)$ is an ordered set where every subset has an infimum and a supremum, then every increasing function has a least fixed point.

\begin{proof}
The proof is that already given earlier
\end{proof}

This means that we can define least fixed-points for higher-order functions as well.
\end{nothing}


\section{Extensions}

\subsection{Monoids}

In essence, all the work we have been doing did not depend, in any essential manner, on taking words to be the underlying monoid. In fact, we can define context-free languages in any monoid:

\lstinputlisting[firstline = 4, lastline = 7]{GeneralizedContextFree.hs}

Kleene star and concatenation is definable in the same way:

\lstinputlisting[firstline = 13, lastline = 19]{GeneralizedContextFree.hs}

This is useful in defining for instance context-free relations, the equivalent of rational relations/finite-state transducers. Here is for instance the duplicating function on strings:

\lstinputlisting[firstline = 22, lastline = 26]{GeneralizedContextFree.hs}

\subsection{Arbitrary structures}

In fact, more generally, context-free languages can be defined on any structure whose primitive operations can be lifted by an Applicative. For instance, trees:

[INSERT EXAMPLE]

What is a context-free language of trees? By an unfortunate accident of history, these are what have been called regular tree languages\ldots

The only crucial piece for us to define a context-free language is the existence of a \textbf{FixedPt} function. \textbf{FixedPt} is only defined for increasing functions. Importantly, lifted operations always result in increasing functions:

\begin{prop}
If $\uparrow$ is a unary operation defined on $S$, then the lifted operation $\uparrow\hspace{-1pt}\uparrow$ on collections of $S$ is increasing in its argument.\\
If $\oplus$ is a binary operation defined on $S$, then the lifted operation $\circledast$ on collections of $S$ is increasing in both arguments.\\
\emph{etc.}
\end{prop}
%





\section*{Proofs} 

\begin{proof}
Because $\mathbf{FixPt}(f)$ is an intersection:
\begin{equation}
\forall L,\ f(L)\subset L \Rightarrow \mathbf{FixPt}(f) \subset L \label{eqn:subset}
\end{equation}
Because $f$ is increasing, we conclude:
$$\forall L,\ f(L)\subset L \Rightarrow f\left(\mathbf{FixPt}(f)\right) \subset f(L)$$
Because $L$ is a subset of $f(L)$:
$$\forall L,\ f(L)\subset L \Rightarrow f\left(\mathbf{FixPt}(f)\right) \subset L$$
So, by intersection:
$$f\left(\mathbf{FixPt}(f)\right) \subset \bigcap \left\lbrace L\ \middle|\ f(L)\subset L\right\rbrace = \mathbf{FixPt}(f)$$
This proves one inclusion. To prove the reverse inclusion, we notice that because $f$ is increasing, we also have
$$f(f(\mathbf{FixPt}(f)))\subset f(\mathbf{FixPt}(f))$$
Plugging in this result in eqn \ref{eqn:subset}, we get:
 $$\mathbf{FixPt}(f) \subset f(\mathbf{FixPt}(f))$$
 So by double inclusion, these two languages are equal. Because all fixed points $L$ are such that $f(L)\subset L$ and $\mathbf{FixPt}(f)$ is the smallest of these languages, $\mathbf{FixPt}(f)$ is the smallest fixed-point
\end{proof}


\begin{proof}
Because $f'$ dominates $f$:
$$f(\mathbf{FixPt}(f'))\subset f'(\mathbf{FixPt}(f')) = \mathbf{FixPt}(f')$$
By definition of $\mathbf{FixPt}(f)$, this entails that
$$\mathbf{FixPt}(f)\subset \mathbf{FixPt}(f')$$
\end{proof}


\begin{proof}
This simply follows from the definition of the lifted version. If $L\subset L'$, then:
$$\uparrow\hspace{-1pt}\uparrow L = \left\lbrace \uparrow x\ \middle|\ x\in L\right\rbrace \subset   \left\lbrace \uparrow x\ \middle|\ x\in L'\right\rbrace = \uparrow\hspace{-1pt}\uparrow L'$$
\end{proof}




\end{document}
